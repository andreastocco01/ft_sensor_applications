In questo capitolo verranno mostrati degli esperimenti per valutare il funzionamento e le prestazioni del sensore. 
Per l'analisi della \textbf{reattivit\`{a}}, viene osservato il comportamento del sensore nel caso in cui ci sia 
un cambiamento istantaneo delle forze in gioco. 
Un'altro importante aspetto da valutare \`{e} la \textbf{precisione} dei dati forniti dal sensore. Per farlo 
si \`{e} pensato di utilizzare le misurazioni effettuate per calcolare la viscosit\`{a} di un liquido di cui se ne conosce 
il valore. 
Prima di mostrare i risultati di questi due esperimenti \`{e} bene, per\`{o}, parlare dell'importanza dell'azzeramento periodico 
del sensore.

\section{Errore nelle misurazioni} \label{sec:drifting}
Come spiegato nel Capitolo \ref{chapter:chapter2} il sensore \'{e} soggetto a rumore di fondo intrinseco, che 
pu\'{o} essere causato da diversi fattori, come la temperatura, la stabilit\'{a} dell'alimentazione o il rumore elettrico. 
\begin{figure}[H]
    \centering
    \includegraphics*[width=0.80\textwidth]{images/drifting.png}
    \caption{Drifting lungo l'asse z}
    \label{fig:drifting}
\end{figure}
In Figura \ref{fig:drifting} viene mostrato il fenomeno del \textbf{drifting}: ossia quando un sensore nel corso del tempo 
mostra una deviazione nelle sue letture senza un'effettiva variazione delle condizioni ambientali. 
Si pu\'{o} notare come la forza rilevata dal sensore lungo l'asse z tenda a crescere col passare del tempo, senza che al sensore 
venga applicata alcuna forza. Gi\'{a} dopo 200 secondi, la forza misurata supera la soglia di confidenza specificata nel manuale 
entro la quale la misurazione deve essere catalogata come non attendibile.
Per ovviare a questo problema \'{e} necessario azzerare il sensore periodicamente, in modo che le letture risultino corrette 
e senza deviazioni.


\section{Analisi della reattivit\`{a}}
In questo esperimento, \'{e} stato attaccato al sensore un filo con appeso un oggetto di 0.25Kg. 
Il braccio \'{e} stato posizionato in modo tale che la forza peso gravasse solo su un asse del sensore alla volta. 
In Figura \ref{boh}, viene mostrato il setup per l'esperimento.
Dopo aver azzerato il sensore, per verificarne la reattivit\'{a}, il filo \'{e} stato tagliato di netto. 
Il taglio del filo \'{e} un ottimo modo per `simulare' un cambiamento di forza istantaneo.
\begin{figure}[H]
    \centering
    \includegraphics*[width=0.80\textwidth]{images/z_cut.png}
    \caption{Andamento taglio del filo lungo l'asse z}
    \label{fig:z_cut}
\end{figure}
In Figura \ref{fig:z_cut} viene mostrato l'andamento della forza rilevata dal sensore lungo l'asse z. 
Si pu\'{o} notare che, fino a quando il filo \'{e} attaccato al sensore, la forza rilevata \'{e} circa zero. 
Questo perch\'{e} il sensore \'{e} stato azzerato quando l'oggetto era gi\'{a} stato appeso. 
Dopo circa 15 secondi, il filo viene tagliato. In questo istante il sensore rileva per una frazione di secondo una forza di circa 1.5N 
(probabilmente dovuta ad un taglio non sufficientemente netto), per poi assestarsi al valore reale della forza rilevata, 
ossia circa -2.5N. 
Tale esperimento \'{e} stato ripetuto anche per gli altri due assi con esiti leggermente migliori. 
I risultati vengono mostrati in Figura \ref{fig:cut_results}.
\begin{figure}[H]
    \centering
    \begin{subfigure}[b]{0.80\textwidth}
        \includegraphics[width=\textwidth]{images/x_cut.png}
        %\caption{}
        \label{fig:x_cut}
    \end{subfigure}
    ~ %add desired spacing between images, e. g. ~, \quad, \qquad, \hfill etc. 
      %(or a blank line to force the subfigure onto a new line)
    \begin{subfigure}[b]{0.80\textwidth}
        \includegraphics[width=\textwidth]{images/y_cut.png}
        %\caption{}
        \label{fig:y_cut}
    \end{subfigure}
    \caption{Andamento esperimento lungo x e y}\label{fig:cut_results}
\end{figure}


\section{Analisi della precisione}
In questa sezione viene mostrato un esperimento per valutare la precisione delle misurazioni del sensore. 
Per farlo si \'{e} pensato di usare i valori delle forze misurate per calcolare la \textbf{viscosit\'{a}} del burro d'arachidi, 
che tipicamente \'{e} compresa tra 1500-2500 $\text{Pa} \cdot \text{s}$. 
A tal proposito \'{e} stato installato sull'UR5 una spatola rettangolare (di dimensioni 6 cm x 4 cm x 4 mm) come end effector. 
La spatola, una volta immersa nel burro di arachidi, verr\'{a} fatta ruotare attorno al suo asse con velocit\'{a} angolare 
costante. Il sensore, rilever\'{a} quindi un momento torcente lungo l'asse z corrispondente alla forza di attrito viscoso esercitata 
dal fluido sulla spatola in rotazione. \'{E} dunque possibile utilizzare tali valori per ricavare sperimentalmente 
la viscosit\'{a} del burro d'arachidi e confrontarla con i dati ufficiali. 
In questo caso, la formula per il calcolo dell'attrito viscoso \'{e} la seguente 
\begin{equation*}
    F = 2 \cdot h \cdot l \cdot \eta \cdot \omega
\end{equation*}
con 
\begin{itemize}
    \item h: altezza della spatola immersa nel fluido
    \item l: larghezza della spatola
    \item $\eta$: viscosit\'{a} del fluido
    \item $\omega$: velocit\'{a} angolare di rotazione
\end{itemize}
Si pu\'{o} quindi `girare' tale formula per ricavare la viscosit\'{a} dalla forza di attrito misurata 
\begin{equation*}
    \eta = \frac{F}{2 \cdot h \cdot l \cdot \omega}
\end{equation*}
In \cite{viscosity} viene mostrato il codice ROS per effettuare l'esperimento.
Inizialmente il braccio viene mosso con MoveIt in modo tale che la spatola entri dentro al burro d'arachidi. 
Successivamente viene cambiato il controllore del robot. Per calcolare la viscosit\'{a} del fluido \'{e} necessario conoscere 
la velocit\'{a} di movimento della spatola. Con un controllore di posizione, tale informazione non \'{e} accessibile. Passando, 
per\'{o}, ad un controllore di velocit\'{a} quale \verb|twist_controller|, la velocit\'{a} di rotazione pu\'{o} essere impostata 
a piacere. Il robot, quindi, comincer\'{a} a far ruotare su se stessa la spatola e il sensore acquisir\'{a} i dati dell'attrito 
viscoso. \'{E} bene notare come la velocit\'{a} di rotazione sia costante, per avere delle misurazioni pi\'{u} uniformi.
Da tutte le misurazioni effettuate dal sensore viene calcolata la media. Poi viene applicata la formula per determinare la 
viscosit'{a}. Il risultato \'{e} $\eta = 1998.0468 \text{Pa} \cdot \text{s}$, che rientra nel range specificato inizialmente.