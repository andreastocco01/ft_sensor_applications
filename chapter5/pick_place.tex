Il pick and place \`{e} un'applicazione largamente utilizzata in ambito industriale e consiste nello spostamento di un oggetto 
da un punto di partenza ad uno di destinazione. Tale compito pu\`{o} essere portato a termine senza l'ausilio di un sensore 
coppia-forza. L'alternativa proposta in questa tesi si concentra su uno scenario di collaborazione uomo-robot, in cui l'operatore 
insegna al robot dove si trova il pezzo da raccogliere e dove dovr\`{a} essere posizionato. 
Per fare ci\`{o} si \`{e} reso necessario installare un gripper come end effector 
dell'UR5 \cite{gripper_repo}. In \cite{environment_setup} con MoveIt Setup Assistant \`{e} stata generata la cartella contenente 
tutti i file di configurazione per questo specifico setup. 
% mettere immagine di RViz
In Figura \ref{fig:pick_place} viene mostrato l'ambiente di lavoro comprensivo del gripper per la presa degli oggetti.
\newpage 
\begin{figure}[H]
    \centering
    \includegraphics*[width=0.65\textwidth]{images/pick_place_setup.jpg}
    \caption{Setup sperimentale dell'applicazione pick and place. L'obiettivo \`{e} inserire gli oggetti sul tavolo 
    all'interno della scatola}
    \label{fig:pick_place}
\end{figure}
Inizialmente il robot apprende il punto di partenza e di destinazione dall'operatore, poi, una volta preso l'oggetto, 
proceder\`{a} con la sua collocazione nella posizione finale.
Di seguito verranno descritte le parti in cui \`{e} stata suddivisa l'applicazione.
\subsection{Salvataggio delle posizioni} \label{sub:positions}
Inizialmente, viene utilizzato l'\textit{inseguitore di forza} per il salvataggio delle posizioni di partenza e di destinazione 
su cui il robot dovr\`{a} spostarsi per portare a termine il proprio compito. 
L'operatore potr\`{a}, quindi, muovere il braccio liberamente fino a quando non si trova al di sopra dell'oggetto da spostare. 
\`{E} stata implementata una gesture che consiste nell'applicazione di una piccola torsione al sensore, 
per salvare la posizione attuale in cui si trova il robot, corrispondente al punto in cui si trova l'oggetto. Il gripper 
si aprir\`{a} e si chiuder\`{a} velocemente per indicare che la posizione \`{e} stata salvata correttamente nel vettore 
\verb|positions|. In Algorithm \ref{algo:positions} viene riportato lo pseudocodice per questa funzionalit\`{a}.
\newpage
\begin{algorithm}[H]
    \caption{Salvataggio posizioni}\label{algo:positions}
    \begin{algorithmic}[1]
        \Require $\text{Forze e momenti rilevati dal sensore}$
        \Ensure $\text{Vettore di posizioni}$
        \State torqueThreshold $\gets$ 1
        \State forces $\gets$ [ ]
        \State positions $\gets$ [ ]
        
        \While{ros::ok() \textbf{and} $\text{positions.size()} < 3$} 
            \If {$\text{forces.size()} < 20$}
                \State forces.pushback(force) // riempio il vettore con le prime 20 misurazioni
            \Else

                // codice inseguitore di forza
                \If{$\text{abs(torque).z} > \text{torqueThreshold}$}
                    \State positions.pushback(currentPosition)
                    \State feedback()
                \EndIf
                \State twistPublisher.publish(twist)
            \EndIf
            \State ros::speedOnce()
        \EndWhile
        \State posPublisher.publish(positions[0])
        \State posPublisher.publish(positions[1])
    \end{algorithmic}
    \end{algorithm}
Lo stesso dovr\`{a} essere fatto anche per la posizione di destinazione e per una posizione ausiliaria in una zona libera 
del piano di lavoro, che verr\`{a} utilizzata per determinare l'altezza del tavolo (vedi \ref{sub:placement}). 
Una volta terminata la fase di acquisizione delle posizioni, 
le prime due verranno pubblicate sul topic \verb|task_positions|.
\subsection{Posizionamento} \label{sub:placement}
In questa fase, il nodo \textbf{placement}\footnotemark{} potr\`{a} iscriversi al topic \verb|task_positions| e leggere le posizioni precedentemente 
acquisiste. Inizialmente, dalla posizione ausiliaria, il robot scender\`{a} verso il basso fino a quando non verr\`{a} rilevata 
una forza lungo l'asse z, corrispondente al contatto con il piano di lavoro. Tale posizione verr\`{a} salvata e utilizzata in seguito 
come valore di riferimento per il calcolo dell'altezza degli oggetti. 
Con MoveIt il braccio viene spostato alla prima posizione del topic, ossia il punto in cui \`{e} presente l'oggetto da prendere.
Da qui, si muover\`{a} verso il basso alla ricerca di un contatto con esso e una nuova posizione verr\`{a} salvata. 
Le differenza delle due posizioni corrisponde all'altezza dell'oggetto. Sar\`{a}, dunque, sufficiente chiudere il gripper 
nel suo punto medio per favorirne una presa pi\`{u} solida. Ad esempio, se l'oggetto \`{e} alto 10 cm, il gripper verr\`{a} chiuso 
ad un'altezza di 5 cm. 
L'UR5, poi, si muover\`{a} nell'altra posizione pubblicata sul topic, ovvero quella di destinazione, 
in attesa di cominciare l'ultima fase dell'applicazione.
\footnotetext{\url{https://github.com/andreastocco01/ur5_ft_tasks/blob/main/src/placement.cpp}}
\subsection{Inserimento} \label{sub:insertion}
Supponendo di voler inserire l'oggetto in un contenitore avente, al centro, una cavit\`{a} di uguale forma e di dimensioni simili, 
non sar\`{a} sufficiente utilizzare la posizione acquisita grossolanamente in \ref{sub:positions}, in quanto troppo poco precisa 
per l'inserimento di oggetti in cavit\`{a} con tolleranze molto strette. 
A tal proposito viene mostrato il codice per calcolare il centro del contenitore (corrispondente al 
centro della cavit\`{a}) in cui inserire l'oggetto\footnotemark{}. Utilizzando il controllore di velocit\`{a} il braccio viene fatto 
muovere da varie posizioni attorno al contenitore verso il contenitore stesso, fino a quando non viene rilevato un contatto con uno 
dei bordi. Il processo viene ripetuto per ciascuno dei quattro bordi e le posizioni di ogni estremit\`{a} vengono salvate per poi essere
utilizzate per calcolare il punto esatto in cui inserire l'oggetto. Il robot potr\`{a}, quindi, posizionarsi al di sopra di 
tali coordinate e cominciare un processo di discesa che lo porter\`{a} ad inserire l'oggetto precisamente nell'alloggiamento finale, 
come mostrato in Figura \ref{fig:insertion}. 
\footnotetext{\url{https://github.com/andreastocco01/ur5_ft_tasks/blob/main/scripts/place_ontop.py}}
\newpage
\begin{figure}[H]
    \centering
    \includegraphics*[width=0.65\textwidth]{images/insertion.jpg}
    \caption{Applicazione pick and place: fase di inserimento dell'oggetto (stella) nella cavità corretta}
    \label{fig:insertion}
\end{figure}
