Come accennato, questa applicazione si focalizza sulla necessit\'{a} di poter muovere il robot a piacimento, senza dover 
ricorrere all'utilizzo del teach pendant o RViz\footnotemark{}. 
Diventa di fondamentale importanza se integrata in applicazioni collaborative, per rendere pi\'{u} user friendly l'interfacciamento 
con il robot. 
Quando il sensore rileva delle forze superiori ad una determinata soglia (calcolata sperimentalmente), viene inviato al robot un 
comando \verb|Twist| per farlo muovere con una velocit\'{a} proporzionale alla forza applicata in input. In questo modo, il robot `segue' 
le forze impartite dall'operatore convertendole in termini di velocit\'{a} di movimento. 
Il codice ROS che implementa tale funzionalit\'{a} viene mostrato in \cite{force_follower}. 
Di seguito viene riportato lo pseudocodice. 
\begin{algorithm}[H]
\caption{Inseguitore di forza}\label{algo:force_follower}
\begin{algorithmic}[1]
    \Require $\text{Forze rilevate dal sensore}$
    \Ensure $\text{Comando Twist}$
    \State alpha $\gets$ 200
    \State threshold $\gets$ 5
    \State forces $\gets$ [ ]
    \State twist $\gets$ 0
    %\State valore\_ritornato $\gets$ \textsc{Funzione}\textsc{Prova}(param1, param2)
    
    \While{ros::ok()} 
    \If {$\text{forces.size()} < 20$}
        \State forces.pushback(force) // riempio il vettore con le prime 20 misurazioni
    \Else
        \State forces.erase(forces.begin()) // rimuovo la misurazione meno recente
        \State forces.pushback(force) // per fare spazio all'ultima rilevata
        \State meanVal $\gets$ mean(forces)
        \State moduleVal $\gets$ module(meanVal) 
        \If {$\text{moduleVal} > threshold$}
            \State twist $\gets$ moduleVal / alpha // scalo il valore del modulo
        \Else
            \State twist $\gets$ 0
        \EndIf
        \State publisher.publish(twist)
    \EndIf
    \State ros::speedOnce()
    \EndWhile
\end{algorithmic}
\end{algorithm}
\footnotetext{\url{https://github.com/andreastocco01/ur5_ft_tasks/blob/main/src/velocity_force_follower.cpp}}
Dopo aver portato il robot in posizione di partenza con MoveIt, viene attivato \verb|twist_controller| con la stessa funzione 
utilizzata nell'esperimento del burro d'arachidi (presente in \verb|utils.cpp|).  
All'interno del vettore \verb|forces| vengono salvate le ultime 20 misurazioni del sensore. Di queste ne viene calcolata la media 
e, se il modulo \'{e} superiore alla soglia specificata, viene inviato al robot un comando \verb|Twist| contenente uno scalamento delle 
forze rilevate nelle tre componenti. Il calcolo della velocit\'{a} di movimento viene fatto sulla media degli ultimi campioni 
per una questione di utilizzabilit\'{a}. Servirsi delle misurazioni singole per il calcolo del vettore velocit\'{a} da inviare al robot 
porta ad un movimento poco fluido e impulsivo che peggiora l'esperienza utente. 
Utilizzando la media, invece, le piccole variazioni 
di forza generate dall'operatore vengono ammortizzate portando ad una risposta da parte del braccio pi\'{u} naturale.
La conversione tra forza e velocit\'{a} avviene mediante un'attenuazione delle componenti del vettore media per un coefficiente 
costante. In questo modo si riduce l'influenza delle forze `piccole' nel vettore velocit\'{a} risultante, valorizzando maggiormente le 
componenti pi\'{u} ampie. 