Una possibile variazione del pick and place consiste nel far si che il braccio \textbf{trovi} la cavit\`{a} in cui inserire l'oggetto. 
In \ref{sub:insertion} si presupponeva che la cavit\`{a} si trovasse al centro del contenitore, in questo modo era possibile, 
determinandone il centro, inserire precisamente l'oggetto nella posizione corretta. Ovviamente se il foro non si trova al centro, 
l'inserimento non andr\`{a} a buon fine. Per risolvere questo problema si \`{e} pensato di sostituire la parte di inserimento 
precedentemente descritta con un nuovo nodo in grado di trovare la posizione del foro\footnotemark{}. 
In Figura \ref{fig:spiral_pick_place} viene mostrato come, in questo caso, il foro in cui inserire l'oggetto non sia pi\`{u} al centro 
del contenitore, bens\`{i} in basso a destra. 
\footnotetext{\url{https://github.com/andreastocco01/ur5_ft_tasks/blob/main/src/spiral_movement.cpp}}
\newpage
\begin{figure}[H]
    \centering
    \includegraphics*[width=0.35\textwidth]{images/spiral_pick_place.jpg}
    \caption{Setup pick and place con movimento a spirale}
    \label{fig:spiral_pick_place}
\end{figure}
Per la ricerca dell'esatta posizione del foro, si assume nota una posizione grossolana da cui partire. 
Il braccio è quindi portato in tale posizione e fatto scendere fino al contatto con il contenitore.
Da qui il robot, comincia ad effettuare un movimento a \textbf{spirale}. 
Mentre effettua tale movimento, mantiene l'oggetto in contatto con la superficie del contenitore. Se il sensore 
non rileva pi\`{u} alcuna forza lungo l'asse z, significa che ci si trova in uno dei seguenti casi:
\begin{itemize}
    \item il contenitore non ha una superficie piana. Il foro non \`{e} ancora stato trovato e quindi \`{e} necessario far scendere 
    il braccio per ristabilire il contatto e continuare a cercare.
    \item il braccio si trova al di sopra del foro. Si pu\`{o} procedere con l'inserimento dell'oggetto.
\end{itemize} 
Per implementare questa funzionalit\`{a} si \`{e} pensato di far scendere il braccio ogni qual volta il sensore non rileva pi\`{u}
una forza lungo l'asse z, come mostrato in Algorithm \ref{algo:spiral}. 
\newpage
\begin{algorithm}[H]
    \caption{Movimento a spirale}\label{algo:spiral}
    \begin{algorithmic}[1]
        \Require $\text{Forze rilevate dal sensore}$
        \Ensure $\text{Inserimento dell'oggetto}$
        \State threshold $\gets$ -8
        \State lim $\gets$ 0.02
        \State twist $\gets$ 0
        \State firstTime $\gets$ true
        \State height
        
        \Repeat 
            \While{$\text{force.z} > \text{threshold}$}
                \State twist $\gets$ (0, 0, -0.015) // vado gi\`{u} finch\`{e} non c'\`{e} un contatto
                \State publish(twist)
            \EndWhile
            \State twist $\gets$ 0
            \State publish(twist) // mi fermo
            \If{firstTime}
                \State heigth $\gets$ currentPosition // salvo l'altezza del contenitore
                \State firstTime $\gets$ false
            \EndIf
            \State currentHeight $\gets$ currentPosition
            \While{$\text{force.z} \leq \text{threshold}$ \textbf{and} $\text{height} - \text{currentHeight} < \text{lim}$}

                // movimento a spirale
            \EndWhile
        \Until{$\text{force.z} > \text{threshold}$ \textbf{and} $\text{height} - \text{currentHeight} < \text{lim}$}
    \end{algorithmic}
    \end{algorithm}
Se la differenza di altezza \`{e} superiore ad una determinata soglia (parametro \textit{lim}), significa che probabilmente si \`{e} 
trovato il foro e quindi il gripper si aprir\`{a} per favorire l'inserimento dell'oggetto. 
A differenza della versione mostrata in \ref{sec:pick_place}, questa non raggiunge sempre l'obiettivo. 
Pu\`{o} capitare, infatti, che il braccio non trovi mai la cavit\`{a} per via dell'incremento del raggio della spirale e che finisca 
al di l\`{a} dei bordi della scatola. Inoltre, se non si trova perfettamente al di sopra del foro, potrebbe non cominciare la 
fase di discesa non portando cos\`{i} a termine il compito. 
In Figura \ref{fig:spiral_insertion}, il quadrato centrale indica il foro in cui inserire l'oggetto, mentre le croci individuano i 
punti da cui \`{e} stato fatto partire il movimento a spirale nei vari test effettuati. 
\begin{figure}[H]
    \centering
    \includegraphics*[width=0.65\textwidth]{images/spiral.png}
    \caption{Test effettuati}
    \label{fig:spiral_insertion}
\end{figure}
Nella Tabella \ref{tab:results} vengono mostrati i risultati nei vari casi
\begin{center}
    \begin{tabular}{ ||c|c|| } 
     \hline
     Test & Esito\\
     \hline\hline 
     (1) & \ding{51} \\ 
     (2) & \ding{51} \\ 
     (3) & \ding{55} \\ 
     (4) & \ding{51} \\ 
     (5) & \ding{55} \\ 
     \hline
    \end{tabular}
    \captionof{table}{Risultati test effettuati \label{tab:results}}
\end{center}
Come gi\`{a} anticipato, con questa applicazione non si ottengono sempre i risultati sperati. In (3) il robot ha cominciato il 
movimento troppo vicino al bordo del contenitore ed \`{e} andato oltre ad esso. In (5), invece, l'interferenza degli altri fori 
ha fatto s\`{i} che si incastrasse senza raggiungere l'obiettivo.
