Una possibile variazione del pick and place consiste nel far si che il braccio \textbf{trovi} la cavit\'{a} in cui inserire l'oggetto. 
In \ref{sub:insertion} si presupponeva che la cavit\'{a} si trovasse al centro del contenitore, in questo modo era possibile, 
determinandone il centro, inserire precisamente l'oggetto nella posizione corretta. Ovviamente se il foro non si trova al centro, 
l'inserimento non andr\'{a} a buon fine. Per risolvere questo problema si \'{e} pensato di sostituire la parte di inserimento 
precedentemente descritta con un nuovo nodo in grado di trovare la posizione del foro \cite{spiral}. 
Quando il braccio si trova in contatto con il contenitore, comincia ad effettuare un movimento a \textbf{spirale}. 
Mentre effettua tale movimento, mantiene l'oggetto in contatto con la superficie del contenitore. Se il sensore 
non rileva pi\'{u} alcuna forza lungo l'asse z, significa che ci si trova in uno dei seguenti casi:
\begin{itemize}
    \item il contenitore non ha una superficie piana. Il foro non \'{e} ancora stato trovato e quindi \'{e} necessario far scendere 
    il braccio per ristabilire il contatto e continuare a cercare.
    \item il braccio si trova al di sopra del foro. Si pu\'{o} procedere con l'inserimento dell'oggetto.
\end{itemize} 
Per implementare questa funzionalit\'{a} si \'{e} pensato di far scendere il braccio ogni qual volta il sensore non rileva pi\'{u}
una forza lungo l'asse z. Se la differenza di altezza \'{e} superiore ad una determinata soglia, significa che probabilmente si \'{e} 
trovato il foro e quindi il gripper si aprir\'{a} per favorire l'inserimento dell'oggetto. 
A differenza della versione mostrata in \ref{sec:pick_place}, questa non raggiunge sempre l'obiettivo. 
Pu\'{o} capitare, infatti, che il braccio non trovi mai la cavit\'{a} per via dell'incremento del raggio della spirale e che finisca 
al di l\'{a} dei bordi della scatola. Inoltre, se non si trova perfettamente al di sopra del foro, potrebbe non cominciare la 
fase di discesa non portando a termine il compito.
% aggiungere tabella esperimenti
% mostrare immagine spirale disegnata
