ROS (Robot Operating System) \`{e} un framework open-source disponibile in Python e C++ per lo sviluppo di applicazioni robotiche. 
Si tratta di un sistema centralizzato che permette alle diverse componenti del sistema (nodi) di comunicare tra loro sia in modo
asincrono (topic) che sincrono (service). Offre inoltre una vasta gamma di strumenti di sviluppo, come un software per la 
visualizzazione grafica (RViz) e la possibilit\`{a} di registrare e riprodurre dati, favorendo cos\`{i} il debugging e il testing 
delle applicazioni. 
La versione raccomandata e utilizzata \`{e} \textbf{ROS Noetic} per \textbf{Ubuntu Focal 20.04}. 
Questo capitolo fornir\`{a} una panoramica esaustiva dei concetti base del Robot Operating System e dell'ambiente di esecuzione 
associato \cite{quigley2009ros}.

\section{Nodo}
Un \textbf{nodo} \`{e} un eseguibile che sfrutta ROS per comunicare con altri nodi. 
A ciascuno di tali eseguibili corrispondono dei file sorgente che, raggruppati in pacchetti, consentono la realizzazione di 
funzionalità comuni (ad esempio un pacchetto con i nodi relativi all'interfacciamento a basso livello col sensore coppia-forza).
Quando viene lanciato il comando \verb|catkin_make|, ogni file sorgente in ogni pacchetto viene compilato 
dando origine ad un nodo.
Impropriamente, si potrebbe quindi dire che un pacchetto \`{e} un insieme di nodi riguardanti la stessa applicazione.
Per poter eseguire un nodo \`{e} sufficiente eseguire il comando \\
\verb|rosrun <nome_pacchetto> <nome_nodo>|. 
Prima di eseguirne uno \`{e} necessario, tuttavia, avviare un ROS Master.
Lo scopo principale di un \textbf{ROS Master} \`{e} quello di consentire ai singoli nodi di localizzarsi a vicenda. Una volta 
fatto partire, essi potranno comunicare tra loro attraverso topic o service.
Per eseguire un ROS Master sar\`{a} sufficiente eseguire il comando \verb|roscore| in un altro terminale.


\section{Topic}
I topic sono dei canali di comunicazione unidirezionali che consentono lo scambio di informazioni tra nodi sottoforma di 
messaggi. Un nodo che pubblica messaggi su un topic viene chiamato publisher, mentre un nodo che legge i messaggi da un topic 
viene chiamato subscriber.
\begin{figure}[H]
    \centering
    \includegraphics*[width=0.75\textwidth]{images/topic_graph.png}
    \caption{Schema di comunicazione}
    \label{fig:topic_graph}
\end{figure}

\section{Service}
I \textbf{service} sono un altro modo in cui i nodi possono comunicare tra loro. A differenza dei topic, che consentono la comunicazione 
asincrona, i service instaurano una comunicazione di tipo `client-server'. Il nodo \textbf{client} invia una richiesta al nodo service e 
attende la sua risposta prima di andare avanti con l'esecuzione. 
Per implementare un service in ROS \'{e} necessario per prima cosa definire la struttura dei messaggi di richiesta e 
risposta \cite{service}. Successivamente il client potr\'{a} creare una richiesta nel formato specificato e inviarla al nodo service  
come parametro della funzione \verb|call()|. 
Il nodo service elaborer\'{a} quindi la richiesta fornendo una risposta al client.

\section{Topic e Service a confronto}
I topic e i service sono due modalit\'{a} di comunicazione molto diverse a livello concettuale.
Con un topic si instaura una comunicazione asincrona in cui tutti i subscriber connessi attendono la pubblicazione 
di un messaggio da parte del publisher, mentre con un service \'{e} il `server' che rimane in attesa di richieste da 
parte dei client connessi. Questo rende i topic pi\'{u} adatti per la trasmissione di flussi di dati continui (come quelli 
provenienti dai sensori) e i service pi\'{u} indicati nel caso di servizi puntuali, come la richiesta di un calcolo o di un'altra 
specifica azione ad un altro nodo \cite{srv_topic_example}.
Nel corso di questa tesi verranno utilizzate entrambe le modalit\'{a} di comunicazione, con prevalenza di quella topic.

\section{Workspace}
Per poter eseguire codice ROS serve un ambiente che permetta l'organizzazione e l'utilizzo di tutti i pacchetti necessari.
Al riguardo, \textbf{catkin} \'{e} il sistema di compilazione ufficiale di ROS che consente la creazione di un workspace per 
organizzare e gestire le applicazioni.
I termini `pacchetto' e `applicazione' sono interscambiabili e possono essere utilizzati in modo equivalente (una definizione pi\'{u} 
rigorosa verr\'{a} data successivamente nel corso di questo capitolo).
Una volta creato il workspace catkin \cite{catkin_ws}, il codice sorgente contenuto all'interno dei pacchetti potr\'{a} essere 
compilato ed eseguito.
