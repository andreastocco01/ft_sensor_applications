ROS (Robot Operating System) \'{e} un framework open-source disponibile in Python e C++ per lo sviluppo di applicazioni robotiche. 
Si tratta di un sistema centralizzato che permette alle diverse componenti del sistema (nodi) di comunicare tra loro sia in modo
asincrono (topic) che sincrono (service). Offre inoltre una vasta gamma di strumenti di sviluppo, come un software per la 
visualizzazione grafica (RViz) e la possibilit\'{a} di registrare e riprodurre dati, favorendo cos\'{i} il debugging e il testing 
delle applicazioni. 
La versione raccomandata e utilizzata \'{e} \textbf{ROS Noetic} per \textbf{Ubuntu Focal 20.04}. 
Questo capitolo ci fornir\'{a} una panoramica esaustiva dei concetti base del Robot Operating System e dell'ambiente di esecuzione 
associato \cite{ros_tutorial}.

\section{Nodo}
Un \textbf{nodo} \'{e} un eseguibile che sfrutta ROS per comunicare con altri nodi.
Quando viene lanciato il comando \verb|catkin_make|, ogni file sorgente in ogni pacchetto viene compilato 
dando origine ad un nodo.
Impropriamente, si potrebbe quindi dire che un pacchetto \'{e} un insieme di nodi riguardanti la stessa applicazione.
Per poter eseguire un nodo \'{e} sufficiente eseguire il comando \\
\verb|rosrun <nome_pacchetto> <nome_nodo>|. 
Prima di eseguirne uno \'{e} necessario, tuttavia, avviare un ROS Master.
Lo scopo principale di un \textbf{ROS Master} \'{e} quello di consentire ai singoli nodi di localizzarsi a vicenda. Una volta 
fatto partire, essi potranno comunicare tra loro attraverso topic o service.
Per eseguire un ROS Master sar\'{a} sufficiente eseguire il comando \verb|roscore| in un altro terminale.


\section{Topic}
I topic sono dei canali di comunicazione unidirezionali che consentono lo scambio di informazioni tra nodi sottoforma di 
messaggi. Un nodo che pubblica messaggi su un topic viene chiamato publisher, mentre un nodo che legge i messaggi da un topic 
viene chiamato subscriber.
\begin{figure}[H]
    \centering
    \includegraphics*[width=0.75\textwidth]{images/topic_graph.png}
    \caption{Schema di comunicazione}
    \label{fig:topic_graph}
\end{figure}
Per pubblicare un messaggio su un topic bisogna utilizzare l'apposita funzione \verb|publish()| passandole come parametro 
il messaggio che vogliamo pubblicare.
A discapito del nome, questa funzione non pubblica effettivamente il messaggio, ma lo mette in una coda d'attesa (la cui 
dimensione viene specificata quando viene istanziato il publisher).
Un thread separato si occupa di inviare effettivamente il messaggio al topic per renderlo visibile a tutti i nodi subscriber 
connessi.
Se il numero di messaggi in coda supera la dimensione di essa, i messaggi pi\'{u} vecchi verranno cancellati per fare spazio 
a quelli pi\'{u} recenti. Quando arriva un nuovo messaggio al subscriber, esso viene salvato 
in una coda d'attesa (stesso funzionamento di quella del publisher) fino a quando ROS non d\'{a} la possibilit\'{a} al nodo 
di eseguire la funzione di callback. Tale funzione \'{e} definita dall'utente e si occupa di processare il messaggio ricevuto. 
ROS eseguir\'{a} una callback solo quando gli daremo il permesso di farlo. Ci sono due modi per fare ci\'{o}:
\begin{itemize}
    \item \verb|ros::spinOnce()| chiede a ROS di eseguire tutte le callback in sospeso e poi ci restituisce il controllo
    \item \verb|ros::spin()| chiede a ROS di attendere e di eseguire tutte le callback in sospeso fino a quando il nodo non 
          viene spento. \'{E} equivalente a:
          \begin{verbatim}
            while (ros::ok()) {
                ros::spinOnce();
            }
          \end{verbatim} 
          \verb|ros::ok()| ritorna 0 quando: 
          \begin{itemize}
            \item il nodo viene spento attraverso un SIGINT (Ctrl-C) oppure dalla chiamata di \verb|ros::shutdown()| in 
                  un altro punto del codice
            \item un altro nodo con lo stesso nome viene eseguito
          \end{itemize}
\end{itemize}
In altre parole \verb|ros::spin()| vincola il nodo a rimanere sempre e solo in attesa di leggere nuovi messaggi. 
Se il nodo non deve solamente eseguire le callback, allora un loop con \verb|ros::spinOnce()| \'{e} la scelta corretta.
(Aggiungere link ad un mio pacchetto di esempio???)


\section{Service}
I \textbf{service} sono un'altra modalit\'{a} di comunicazione tra nodi. A differenza dei topic, che consentono la comunicazione 
asincrona, i service instaurano una comunicazione di tipo `client-server'. Il nodo \textbf{client} invia una richiesta al nodo service e 
attende la sua risposta prima di andare avanti con l'esecuzione. 
Per implementare un service in ROS \'{e} necessario per prima cosa definire la struttura dei messaggi di richiesta e 
risposta \cite{service}. Successivamente il client potr\'{a} creare una richiesta nel formato specificato e inviarla al nodo service  
come parametro della funzione \verb|call()|. 
Il nodo service elaborer\'{a} quindi la richiesta fornendo una risposta al client.

\section{Topic e Service a confronto}
I topic e i service sono due modalit\`{a} di comunicazione molto diverse a livello concettuale.
Con un topic si instaura una comunicazione asincrona in cui tutti i subscriber connessi attendono la pubblicazione 
di un messaggio da parte del publisher, mentre con un service \`{e} il `server' che rimane in attesa di richieste da 
parte dei client connessi. Questo rende i topic pi\`{u} adatti per la trasmissione di flussi di dati continui (come quelli 
provenienti dai sensori) e i service pi\`{u} indicati nel caso di servizi puntuali, come la richiesta di un calcolo o di un'altra 
specifica azione ad un altro nodo \cite{srv_topic_example}.
Nel corso di questa tesi verranno utilizzate entrambe le modalit\`{a} di comunicazione, con prevalenza di quella topic.

\section{Workspace}
Per poter eseguire codice ROS abbiamo bisogno di un ambiente che permetta l'organizzazione e l'utilizzo di tutti i nostri pacchetti.
Catkin \'{e} il sistema di compilazione ufficiale di ROS che ci consente di creare un workspace per organizzare 
e gestire le nostre applicazioni.
I termini `pacchetto' e `applicazione' sono interscambiabili e possono essere utilizzati in modo equivalente (una definizione pi\'{u} 
rigorosa verr\'{a} data nel corso di questo capitolo).
Una volta creato il workspace catkin \cite{catkin_ws}, saremo in grado di compilare il codice sorgente contenuto all'interno 
dei nostri pacchetti.
