Gli esperimenti condotti in questa tesi hanno dimostrato che il sensore coppia-forza \`{e} uno strumento affidabile e preciso. 
Prima ne \`{e} stata valutata la reattivit\`{a} attraverso il cambiamento istantaneo delle forze applicate ad esso, 
successivamente ne \`{e} stata analizzata la precisione, utilizzando le piccole forze rilevate per calcolare la viscosit\`{a} del 
burro d'arachidi. 
In entrambi i casi, i risultati ottenuti sono ripetibili e coerenti con gli obiettivi specificati inizialmente. 
Sono state, poi, mostrate diverse applicazioni molto diffuse in letteratura come, ad esempio, il pick and place anche se, per 
quest'ultima, non sarebbe necessario disporre di un sensore coppia-forza. La tesi ha mostrato come l'interpretazione 
delle forze applicate al braccio possano incrementare la versatilit\`{a} e l'efficacia dell'applicazione nonch\`{e} l'esperienza 
utente. Utilizzando l'inseguitore di forza \`{e} possibile istruire il robot sulle posizioni variabili degli oggetti e, 
facendo una stima sull'altezza di essi, il braccio potr\`{a} effettuare una presa pi\`{u} solida e precisa. 
\`{E} stata presentata anche una variazione, basata su un movimento a spirale, nel caso in cui non sia nota precisamente la posizione 
obiettivo. Utilizzando i dati 
del sensore si riesce a dedurre se ci si trova o meno al di sopra della cavit\`{a} in cui inserire l'oggetto. 
\`{E} stato, inoltre, presentato un esempio di trasporto collaborativo, con il quale il robot aiuta l'operatore nel trasporto 
di un materiale sottile e leggero. 
La tesi ha delineato alcune possibili direzioni per migliorare ulteriormente le prestazioni delle applicazioni presentate. 
Ci\`{o} include, ad esempio, l'integrazione di telecamere e l'intelligenza artificiale per migliorare ulteriormente la capacit\`{a} 
di percezione e controllo del robot. 
In conclusione, la tesi ha confermato l'importanza e l'utilità del sensore coppia-forza nelle applicazioni robotiche, 
offrendo spunti per il suo miglioramento e indicando possibili sviluppi futuri. L'utilizzo dei sensori coppia-forza apre 
nuove possibilità nel controllo e nella manipolazione degli oggetti, contribuendo alla crescita e all'evoluzione 
della robotica industriale.