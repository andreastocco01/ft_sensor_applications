I robot manipolatori hanno rivoluzionato l'automazione industriale, permettendo lo svolgimento
di operazioni complesse in modo rapido, preciso e sicuro.
Un ruolo chiave nel controllo dei manipolatori viene assunto dai sensori coppia-forza,
che permettono di misurare e regolare la forza esercitata dal robot durante lo svolgimento delle sue attivit\'{a}.
L'UR5 di Universal Robot \'{e} un modello ampiamente utilizzato per via della sua flessibilit\'{a} ed efficienza.
Tuttavia, a differenza di altri modelli, l'UR5 non \'{e} dotato di sensori coppia-forza.
Sar\'{a} dunque necessario installarne uno manualmente. A tale scopo si \'{e} scelto di utilizzare l'FT 300-S
di Robotiq.
L'obiettivo di questa tesi \'{e} quello di implementare un sistema di controllo della forza per l'UR5 utilizzando i dati
forniti dall'FT 300-S e il framework di sviluppo ROS (Robot Operating System).
ROS \'{e} un sistema ampiamente utilizzato nella comunit\'{a} informatica perch\'{e} fornisce strumenti e librerie 
per il controllo e la comunicazione tra le componenti di un sistema robotico.
Verranno effettuati alcuni test per valutare le prestazioni del sensore in termini di reattivit\'{a} e precisione.
Successivamente verranno presentate delle possibili applicazioni volte a dimostrare l'efficacia
di tali sensori per lo svolgimento di attivit\'{a} industriali.
