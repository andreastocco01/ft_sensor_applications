I robot manipolatori hanno rivoluzionato l'automazione industriale, permettendo lo svolgimento
di operazioni complesse in modo rapido, preciso e sicuro.
Un ruolo chiave nel controllo di questi robot viene assunto dai sensori coppia-forza,
che permettono di misurare e regolare la forza esercitata dal robot durante lo svolgimento delle proprie attivit\'{a}. 
Uno dei modelli di robot manipolatori pi\'{u} utilizzato \'{e} l'\textbf{UR5} di \textbf{Universal Robot}, per via della sua flessibilit\'{a} 
ed efficienza.
Tuttavia, a differenza di altri, l'UR5 non \'{e} dotato di sensori coppia-forza.
\'{E} stato, dunque, necessario installarne uno manualmente. A tale scopo si \'{e} scelto di utilizzare il sensore \textbf{FT 300-S} 
di \textbf{Robotiq}. 
In questa tesi verr\'{a} presentata l'implementazione di un sistema di controllo della forza per l'UR5 utilizzando i dati
forniti dall'FT 300-S e il framework di sviluppo \textbf{ROS (Robot Operating System)}.
ROS \'{e} ampiamente utilizzato dalla comunit\'{a} informatica perch\'{e} fornisce strumenti e librerie 
per il controllo e la comunicazione tra le componenti di un sistema robotico.
Verranno effettuati prima, alcuni test per valutare le prestazioni del sensore in termini di reattivit\'{a} e precisione, 
e successivamente presentate delle possibili applicazioni volte a dimostrare l'efficacia
di tali sensori per lo svolgimento di attivit\'{a} industriali.
% aggiungere il fatto che si e' scelto di scrivere il codice in C++
