Il sensore FT300-S di Robotiq \'{e} stato installato all'estremit\'{a} dell'UR5 e collegato alla control box per ricevere 
l'alimentazione necessaria. 
\begin{figure}[H]
    \centering
    \includegraphics*[width=0.5\textwidth]{images/ft.png}
    \caption{FT300-S}
    \label{fig:ft}
\end{figure}
\'{E} importante notare come la sua presenza non precluda la possibilit\'{a} di installazione di un end effector, 
che pu\'{o} essere facilmente posizionato `al di sopra' del sensore. 
L'FT300-S \'{e} in grado di rilevare forze e torsioni nel range di $\pm 300 N$ e $\pm 30 Nm$ rispettivamente. 
Le misurazioni del sensore hanno un rumore di fondo intrinseco, \'{e} quindi necessario scartare tutti i dati al di sotto delle 
soglie consigliate nel manuale \cite{ft_sensor} in quanto non attendibili. 
Per collegare il sensore al PC sono state provate due alternative: 
\begin{itemize}
    \item collegamento via USB tra sensore e control box e via ethernet tra control box e computer
    \item collegamento diretto via USB tra sensore e computer
\end{itemize}

\subsection{Collegamento via USB tra sensore e control box e via ethernet tra control box e computer}
\begin{figure}[H]
    \centering
    \includegraphics*[width=0.1\textwidth]{images/ft-cbox-pc.png}
    \caption{Schema collegamento}
    \label{fig:ft-cbox-pc}
\end{figure}
bla bla bla

\subsection{Collegamento diretto via USB tra sensore e computer}
\begin{figure}[H]
    \centering
    \includegraphics*[width=0.1\textwidth]{images/ft-pc.png}
    \caption{Schema collegamento}
    \label{fig:ft-pc}
\end{figure}
bla bla bla

