Dopo aver introdotto i concetti base di ROS e descritto le principali caratteristiche di robot e sensore, in questa sezione si 
parler\'{a} di \textbf{MoveIt}. 
MoveIt \'{e} un framework specifico di ROS specializzato nella \textbf{pianificazione del movimento}. Offre un'ampia gamma 
di strumenti e librerie per la generazione delle traiettorie, la gestione della cinematica, la simulazione e il controllo 
dei robot. 
Prima di procedere, \'{e} per\'{o} opportuno fornire una breve introduzione ai file URDF.
Gli \textbf{URDF (Unified Robot Description Format)} sono dei file basati sul formato XML (eXtensible Markup Language) e 
costituiscono uno standard per rappresentare la geometria, la cinematica e altre caratteristiche dei robot all'interno di ROS. 
Grazie a questi file, è possibile definire la gerarchia dei \textbf{link} del robot, specificandone anche informazioni quali  
le dimensioni, la massa e l'inerzia. Inoltre, gli URDF consentono di modellare i \textbf{giunti}
del robot, definendone i limiti di movimento e le relazioni cinematiche con i link adiacenti. 
Robotiq e Universal Robot mettono a disposizione i file URDF dei propri prodotti. Per ricreare l'ambiente di lavoro presente in 
laboratorio \'{e} stato necessario `unire' la rappresentazione del robot con quella del sensore in un nuovo file URDF contenente 
anche caratteristiche proprie dell'ambiente, come il tavolo su cui \'{e} montato il braccio, il piano di lavoro e il muro. 
In Figura \ref{fig:workcell} viene mostrato l'ambiente di lavoro in cui sono state provate le applicazioni proposte. 
\textbf{RViz (ROS Visualization)} \'{e} uno strumento di visualizzazione 3D incluso in ROS che, oltre a consentire la visualizzazione 
dei movimenti del robot, offre altri strumenti per interagire con esso.  
Per semplificare il processo di configurazione e setup del sistema, MoveIt mette a disposizione il \textbf{MoveIt Setup Assistant},  
un software che fornisce un'interfaccia grafica per permettere agli utenti di generare i file di configurazione necessari 
all'utilizzo di MoveIt. 
In \cite{environment_setup} la cartella \verb|ur5_ft_moveit_config| \'{e} stata generata da MoveIt Setup Assistant a partire 
dal file \verb|ur5_ft.urdf.xacro| (contenente la descrizione dell'ambiente) presente all'interno di \verb|environment_description|. 
In \verb|environment_manager| sono presenti dei \textbf{launch file} (file XML per l'avvio simultaneo di pi\'{u} nodi, che 
permettono la definizione e il passaggio di parametri tra di essi) che a cascata fanno partire nodi per il collegamento remoto del 
PC al robot, l'avvio di MoveIt e Rviz e l'inizializzazione del sensore. 
Sar\'{a} dunque sufficiente eseguire il comando \verb|roslaunch environment_manager ur5_ft_load_all| per essere poi in grado di 
eseguire le applicazioni proposte nel Capitolo \ref{chapter:chapter5}. 
\begin{figure}[H]
    \centering
    \includegraphics*[width=0.70\textwidth]{images/workcell.png}
    \caption{Simulazione dell'ambiente di lavoro su RViz}
    \label{fig:workcell}
\end{figure}