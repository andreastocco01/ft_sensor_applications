L'UR5 possiede diversi \textbf{controllori} integrati per gestire il movimento e il funzionamento del robot. 
\begin{figure}[H]
    \centering
    \includegraphics*[width=0.43\textwidth]{images/controller_manager.png}
    \caption{Lista controllori UR5}
    \label{fig:controllers}
\end{figure}
In Figura \ref{fig:controllers}, viene mostrata la lista dei controllori disponibili per manovrare l'UR5 insieme ai rispettivi 
stati di funzionamento. 
I controllori segnati in verde sono Read-Only controllers, ossia dei controllori che leggono solamente lo stato attuale del 
robot e lo pubblicano su un topic. Non c'\'{e} nessuna limitazione sul numero di controllori Read-Only in esecuzione 
nello stesso momento. Gli altri, invece, sono Commanding controllers, ossia controllori che consentono l'alterazione dello 
stato del robot. Non \'{e} possibile utilizzare pi\'{u} controllori di questo tipo contemporaneamente. 
MoveIt permette un controllo di tipo \textbf{posizionale}, ossia, internamente, calcola la traiettoria migliore 
per andare da un punto di partenza nello spazio ad un punto di arrivo. 
Una volta calcolata la traiettoria, con l'ausilio di \verb|scaled_pos_joint_traj_controller| 
vengono modificati i valori dei giunti per consentire al robot di seguire la traiettoria specificata. 
Nelle applicazioni mostrate nel Capitolo \ref{chapter:chapter4} viene utilizzato anche il controllore di 
\textbf{velocit\'{a}} \verb|twist_controller|. Questo controllore riceve un comando di twist come input, 
che specifica la velocit\'{a} di traslazione e rotazione lungo gli assi x, y e z. 
Il controllore, poi, lo traduce in comandi di controllo per i motori del robot.
Essendo entrambi Commanding controllers, l'utilizzo di uno esclude l'utilizzo 
dell'altro. ROS mette a disposizione dei service per gestire i controllori attivi del robot. Ad esempio, 
\verb|/controller_manager/switch_controller| si occupa di scambiare lo stato di esecuzione dei due controllori che riceve 
in input \cite{controller_manager}. Tali service risultano, quindi, molto utili nelle applicazioni proposte perch\'{e} permettono 
l'utilizzo del controllore pi\'{u} appropriato in base alle specifiche esigenze.