Come accennato, questa applicazione si focalizza sulla necessit\'{a} di poter muovere il robot a piacimento, senza dover 
ricorrere all'utilizzo del teach pendant o RViz. 
Diventa di fondamentale importanza se integrata in applicazioni collaborative, per rendere pi\'{u} user friendly l'interfacciamento 
con il robot. 
Quando il sensore rileva delle forze superiori ad una determinata soglia (calcolata sperimentalmente), viene inviato al robot un 
comando \verb|Twist| per farlo muovere con una velocit\'{a} proporzionale alla forza applicata in input. In questo modo, il robot `segue' 
le forze impartite dall'operatore convertendole in termini di velocit\'{a} di movimento. 
Il codice ROS che implementa tale funzionalit\'{a} viene mostrato in \cite{force_follower}. 
Dopo aver portato il robot in posizione di partenza con MoveIt, viene attivato \verb|twist_controller| con la stessa funzione 
utilizzata nell'esperimento del burro d'arachidi (presente in \verb|utils.cpp|). 
Per il momento non \'{e} necessario focalizzarsi sulla funzione \verb|feedback()|, sul publisher \verb|gripper_publisher| e nemmeno 
sul vettore \verb|positions|, in quanto sono elementi che verranno utilizzati nell'applicazione \textbf{pick and place}. 
All'interno del vettore \verb|forces| vengono salvate le ultime 20 misurazioni del sensore. Di queste ne viene calcolata la media 
e, se il modulo \'{e} superiore alla soglia specificata, viene inviato al robot un comando \verb|Twist| contenente uno scalamento delle 
forze rilevate nelle tre componenti. Il calcolo della velocit\'{a} di movimento viene fatto sulla media degli ultimi campioni 
per una questione di utilizzabilit\'{a}. Servirsi delle misurazioni singole per il calcolo del vettore velocit\'{a} da inviare al robot 
porta ad un movimento poco fluido e impulsivo che peggiora l'esperienza utente. 
La conversione tra forza e velocit\'{a} avviene mediante un'attenuazione delle componenti del vettore media per un coefficiente 
costante. Questo per ridurre l'influenza delle forze `piccole' nel vettore velocit\'{a} risultante, valorizzando maggiormente le 
componenti pi\'{u} ampie. 