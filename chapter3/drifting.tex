Come spiegato nel Capitolo \ref{chapter:chapter2} il sensore \'{e} soggetto a rumore di fondo intrinseco, che 
pu\'{o} essere causato da diversi fattori, come la temperatura, la stabilit\'{a} dell'alimentazione o il rumore elettrico. 
\begin{figure}[H]
    \centering
    \includegraphics*[width=0.80\textwidth]{images/drifting.png}
    \caption{Drifting lungo l'asse z}
    \label{fig:drifting}
\end{figure}
In Figura \ref{fig:drifting} viene mostrato il fenomeno del \textbf{drifting}: ossia quando un sensore nel corso del tempo 
mostra una deviazione nelle sue letture senza un'effettiva variazione delle condizioni ambientali. 
Si pu\'{o} notare come la forza rilevata dal sensore lungo l'asse z tenda a crescere col passare del tempo, senza che al sensore 
venga applicata alcuna forza. Gi\'{a} dopo 200 secondi, la forza misurata supera la soglia di confidenza specificata nel manuale 
entro la quale la misurazione deve essere catalogata come non attendibile.
Per ovviare a questo problema \'{e} necessario azzerare il sensore periodicamente, in modo che le letture risultino corrette 
e senza deviazioni.
