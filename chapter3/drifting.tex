Come spiegato nel Capitolo \ref{chapter:chapter2} il sensore \`{e} soggetto a rumore di fondo intrinseco, che 
pu\`{o} essere causato da diversi fattori, come la temperatura, l'instabilit\`{a} dell'alimentazione o il rumore elettrico. 
\begin{figure}[H]
    \centering
    \includegraphics*[width=0.80\textwidth]{images/drifting.png}
    \caption{Drifting lungo l'asse z}
    \label{fig:drifting}
\end{figure}
In Figura \ref{fig:drifting} viene mostrato il fenomeno del \textbf{drifting}: ossia quando un sensore nel corso del tempo 
mostra una deviazione nelle sue letture senza un'effettiva variazione delle condizioni ambientali. Questo fenomeno si verifica 
solamente se il sensore viene collegato al PC come indicato nella Sezione \ref{sec:scp}. 
Si pu\`{o} notare come la forza rilevata dal sensore lungo l'asse z tenda a crescere col passare del tempo, senza che al sensore 
venga applicata alcuna forza. Gi\`{a} dopo 200 secondi, la forza misurata supera la soglia di confidenza specificata nel manuale 
entro la quale la misurazione deve essere catalogata come non attendibile.
Per ovviare a questo problema, nel caso in cui si optasse per un collegamento passante per la control box del robot, 
sarebbe necessario azzerare il sensore periodicamente in modo che le letture risultino corrette e senza deviazioni. 
Gli esperimenti descritti nel corso di questo capitolo e le applicazioni presenti nel Capitolo \ref{chapter:chapter5}, sono state 
sviluppate utilizzando un collegamento diretto tra sensore e PC via USB.
