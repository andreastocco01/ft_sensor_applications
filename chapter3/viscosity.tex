In questa sezione viene mostrato un esperimento per valutare la precisione delle misurazioni del sensore\footnotemark{}. 
Per farlo si \`{e} pensato di usare i momenti angolari misurati dal sensore per calcolare la \textbf{viscosit\`{a}} del burro 
d'arachidi e confrontarla con il valore ufficiale noto (150-250 $\text{Pa} \cdot \text{s}$). 
A tal proposito \`{e} stato installato sull'UR5 un cilindro (di raggio 2cm e altezza 7.5cm) come end effector. 
Si \`{e} scelto di utilizzare un tool con questa forma per agevolare la determinazione della formula dell'attrito viscoso agente 
in questo caso.
In Figura \ref{fig:peanut_butter} viene mostrato il setup per questo esperimento. 
\newpage
\begin{figure}[H]
    \centering
    \includegraphics*[width=0.45\textwidth]{images/setup_viscosity.jpg}
    \caption{Setup sperimentale per l'analisi della precisione, composto da un vasetto di burro d'arachidi e un tool 
    cilindrico stampato in 3D}
    \label{fig:peanut_butter}
\end{figure}
Il cilindro, una volta immerso nel burro di arachidi, viene fatto ruotare attorno al suo asse con velocit\`{a} angolare 
costante. Il sensore, rileva quindi un momento torcente lungo l'asse z corrispondente alla forza di attrito viscoso esercitata 
dal fluido sul cilindro in rotazione. \`{E} dunque possibile utilizzare tali valori per ricavare sperimentalmente 
la viscosit\`{a} del burro d'arachidi e confrontarla con i dati ufficiali noti. 
La formula per il calcolo dell'attrito viscoso \`{e} la seguente 
\begin{equation*}
    M = 2\pi \cdot \eta \cdot \frac{\omega \cdot h}{\Delta r} \cdot r_{1}^{3}
\end{equation*}
con 
\begin{itemize}
    \item $\eta$: viscosit\`{a} del fluido
    \item $\omega$: velocit\`{a} angolare di rotazione
    \item h: altezza del cilindro
    \item $r_{1}$: raggio del cilindro
    \item $r_{2}$: raggio del barattolo
    \item $\Delta r$: $r_{2} - r_{1}$
\end{itemize}
Si pu\`{o} quindi `ribaltare' tale formula per ricavare la viscosit\`{a} dalla forza di attrito misurata 
\begin{equation} \label{eq:eta}
    \eta = \frac{M \cdot \Delta r}{2\pi \cdot \omega \cdot h \cdot r_{1}^{3}}
\end{equation}
Utilizzando velocit\`{a} di rotazione e dimensioni del cilindro ridotte, la forza d'attrito rilevata dal sensore sar\`{a} in generale 
piccola. \`{E} stato necessario utilizzare un liquido con una viscosit\`{a} elevata, quale il burro d'arachidi, per condurre l'esperimento in modo tale 
che le misurazioni non fossero minori della sensibilit\`{a} del sensore. 
Con MoveIt il braccio viene prima posizionato in modo tale che il cilindro sia immerso all'interno del burro d'arachidi. 
Come indicato in \ref{eq:eta}, per calcolare la viscosit\`{a} del fluido \`{e} necessario conoscere 
la velocit\`{a} di rotazione del cilindro. Con un controllore di posizione, tale informazione non \`{e} accessibile. 
Pertanto \`{e} necessario passare ad un controllore di velocit\`{a} (quale \verb|twist_controller|) per poter manovrare il robot in 
termini di velocit\`{a} e non in termini di posizione. Essendo questa un'operazione effettuata anche in altre applicazioni (vedi  
Capitolo \ref{chapter:chapter4}), sono state create delle apposite funzioni per il cambio dei controllori nel file \verb|utils.cpp|. 
Il robot, quindi, comincia a far ruotare su se stesso il cilindro a velocit\`{a} costante ($0.8 \frac{\text{rad}}{\text{s}}$) mentre 
il sensore acquisisce i dati della forza d'attrito. 
Viene poi calcolata la media di tutte le misurazioni effettuate (600) e, tale valore, viene usato per calcolare la viscosit\`{a} 
del burro d'arachidi. 
Nella seguente tabella vengono riportati i risultati delle misurazioni effettuate (in ordine cronologico)
\begin{center}
    \begin{tabular}{ ||c|c|c|c|c|c|| } 
     \hline
     $M \left[N \cdot m\right]$ & $\Delta r \left[cm\right]$ & $\omega \left[\frac{rad}{s}\right]$ & $h \left[cm\right]$ & $r_{1} \left[cm\right]$ & $\eta \left[Pa \cdot s\right]$\\
     \hline\hline 
     0.037298 & 1.5 & 0.8 & 7.5 & 2 & 185.506634 \\ 
     0.033253 & 1.5 & 0.8 & 7.5 & 2 & 165.388578 \\ 
     0.030207 & 1.5 & 0.8 & 7.5 & 2 & 150.235578 \\ 
     0.030003 & 1.5 & 0.8 & 7.5 & 2 & 149.224396 \\ 
     0.029157 & 1.5 & 0.8 & 7.5 & 2 & 145.013306 \\ 
     0.028872 & 1.5 & 0.8 & 7.5 & 2 & 143.595900 \\ 
     \hline
    \end{tabular}
\end{center}
La viscosit\`{a} fluttua parecchio nelle diverse misurazioni. Questo perch\'{e} le caratteristiche fisiche del burro d'arachidi 
cambiano durante ogni esperimento (aumento della temperatura per via dello sfregamento). Per avere misurazioni pi\`{u} attendibili 
sarebbe stato necessario riportare il fluido in condizione di riposo al termine di ogni misurazione. Il risultato pi\`{u} affidabile 
\`{e} il primo, in quanto \`{e} stato effettuato quando le propriet\`{a} fisiche del burro d'arachidi non erano ancora 
state alterate.
In media, la viscosit\`{a} del burro d'arachidi \`{e} $\eta = 156.494065$ $\text{Pa} \cdot \text{s}$, che rientra nel range di 
valori noto specificato inizialmente. 
\footnotetext{\url{https://github.com/andreastocco01/ur5_ft_tasks/blob/main/src/viscosity.cpp}}