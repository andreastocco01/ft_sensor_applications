In questa sezione viene mostrato un esperimento per valutare la precisione delle misurazioni del sensore. 
Per farlo si \'{e} pensato di usare i valori delle forze misurate per calcolare la \textbf{viscosit\'{a}} del burro d'arachidi, 
che tipicamente \'{e} compresa tra 1500-2500 $\text{Pa} \cdot \text{s}$. 
A tal proposito \'{e} stato installato sull'UR5 una spatola rettangolare (di dimensioni 6 cm x 4 cm x 4 mm) come end effector. 
La spatola, una volta immersa nel burro di arachidi, verr\'{a} fatta ruotare attorno al suo asse con velocit\'{a} angolare 
costante. Il sensore, rilever\'{a} quindi un momento torcente lungo l'asse z corrispondente alla forza di attrito viscoso esercitata 
dal fluido sulla spatola in rotazione. \'{E} dunque possibile utilizzare tali valori per ricavare sperimentalmente 
la viscosit\'{a} del burro d'arachidi e confrontarla con i dati ufficiali. 
In questo caso, la formula per il calcolo dell'attrito viscoso \'{e} la seguente 
\begin{equation*}
    F = 2 \cdot h \cdot l \cdot \eta \cdot \omega
\end{equation*}
con 
\begin{itemize}
    \item h: altezza della spatola immersa nel fluido
    \item l: larghezza della spatola
    \item $\eta$: viscosit\'{a} del fluido
    \item $\omega$: velocit\'{a} angolare di rotazione
\end{itemize}
Si pu\'{o} quindi `girare' tale formula per ricavare la viscosit\'{a} dalla forza di attrito misurata 
\begin{equation*}
    \eta = \frac{F}{2 \cdot h \cdot l \cdot \omega}
\end{equation*}
In \cite{viscosity} viene mostrato il codice ROS per effettuare l'esperimento.
Inizialmente il braccio viene mosso con MoveIt in modo tale che la spatola entri dentro al burro d'arachidi. 
Successivamente viene cambiato il controllore del robot. Per calcolare la viscosit\'{a} del fluido \'{e} necessario conoscere 
la velocit\'{a} di movimento della spatola. Con un controllore di posizione, tale informazione non \'{e} accessibile. Passando, 
per\'{o}, ad un controllore di velocit\'{a} quale \verb|twist_controller|, la velocit\'{a} di rotazione pu\'{o} essere impostata 
a piacere. Il robot, quindi, comincer\'{a} a far ruotare su se stessa la spatola e il sensore acquisir\'{a} i dati dell'attrito 
viscoso. \'{E} bene notare come la velocit\'{a} di rotazione sia costante, per avere delle misurazioni pi\'{u} uniformi.
Da tutte le misurazioni effettuate dal sensore viene calcolata la media. Poi viene applicata la formula per determinare la 
viscosit'{a}. Il risultato \'{e} $\eta = 1998.0468 \text{Pa} \cdot \text{s}$, che rientra nel range specificato inizialmente.