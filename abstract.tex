%definisco il layout dell'abstract
\def\changemargin#1#2{\list{}{\rightmargin#2\leftmargin#1}\item[]}
\let\endchangemargin=\endlist

%Genero l'ambiente per l'abstract
\newcommand\summaryname{Abstract}
\newenvironment{Abstract}%
    {\begin{center}%
    \bfseries{\summaryname} \end{center}}

\begin{Abstract}
%\begin{changemargin}{1cm}{1cm}
    I sensori coppia-forza sono componenti fondamentali nei sistemi robotici in quanto forniscono dati sulle forze e i momenti 
    esterni applicati al robot. Questi dati possono essere utilizzati per applicazioni a supporto della collaborazione uomo-robot 
    e per l'automatizzazione di attivit\`{a} che richiedono elevata precisione. ROS (Robot Operating System) \`{e} un framework che 
    fornisce una vasta gamma di librerie e strumenti software per lo sviluppo di applicazioni robotiche. In questa tesi verranno 
    mostrate delle possibili applicazioni ROS in ambito industriale per i sensori coppia-forza, previa verifica della loro accuratezza 
    in termini di reattivit\`{a} e precisione. Questa verifica sar\`{a} effettuata attraverso due esperimenti: il primo riguardante il 
    taglio di un filo a cui \`{e} attaccato un peso e il secondo relativo al calcolo della viscosit\`{a} di un fluido. 
    I risultati ottenuti da tali esperimenti forniranno una solida base di validazione per l'utilizzo dei sensori coppia-forza 
    nelle applicazioni industriali, dimostrando la loro capacit\`{a} di rispondere tempestivamente ai cambiamenti delle forze in gioco 
    e di fornire misurazioni precise.
%\end{changemargin}
\end{Abstract}