Un \textbf{nodo} \`{e} un eseguibile che sfrutta ROS per comunicare con altri nodi. 
A ciascuno di tali eseguibili corrispondono dei file sorgente che, raggruppati in pacchetti, consentono la realizzazione di 
funzionalità comuni (ad esempio un pacchetto con i nodi relativi all'interfacciamento a basso livello col sensore coppia-forza).
Quando viene lanciato il comando \verb|catkin_make|, ogni file sorgente in ogni pacchetto viene compilato 
dando origine ad un nodo.
Impropriamente, si potrebbe quindi dire che un pacchetto \`{e} un insieme di nodi riguardanti la stessa applicazione.
Per poter eseguire un nodo \`{e} sufficiente eseguire il comando \\
\verb|rosrun <nome_pacchetto> <nome_nodo>|. 
Prima di eseguirne uno \`{e} necessario, tuttavia, avviare un ROS Master.
Lo scopo principale di un \textbf{ROS Master} \`{e} quello di consentire ai singoli nodi di localizzarsi a vicenda. Una volta 
fatto partire, essi potranno comunicare tra loro attraverso topic o service.
Per eseguire un ROS Master sar\`{a} sufficiente eseguire il comando \verb|roscore| in un altro terminale.
