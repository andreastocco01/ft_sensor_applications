Un pacchetto ROS contiene al suo interno codici sorgenti, librerie e dati di configurazione.
Per creare un pacchetto \'{e} sufficiente eseguire il comando \\
\verb|catkin_create_pkg <nome_pacchetto>| all'interno della cartella \verb|src/| situata nel workspace catkin.
Un nodo \'{e} un eseguibile che sfrutta ROS per comunicare con altri nodi.
Quando viene lanciato il comando \verb|catkin_make|, ogni file sorgente in ogni pacchetto viene compilato 
dando origine ad un nodo.
Impropriamente, si potrebbe quindi dire che un pacchetto \'{e} un insieme di nodi riguardanti la stessa applicazione.
Per poter eseguire un nodo \'{e} sufficiente eseguire il comando \verb|rosrun <nome_pacchetto> <nome_nodo>|.
Tuttavia si incorrer\'{a} in un errore se prima non viene fatto partire un ROS Master.
Lo scopo principale di un ROS Master \'{e} quello di consentire ai singoli nodi di localizzarsi a vicenda. Una volta 
fatto questo, essi potranno comunicare tra loro attraverso topic o service.
Per far partire un ROS Master sar\'{a} sufficiente eseguire il comando \verb|roscore| in un altro terminale.
