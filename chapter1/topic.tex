I \textbf{topic} sono dei canali di comunicazione unidirezionali che consentono lo scambio di informazioni tra nodi sottoforma di 
messaggi. Un nodo che pubblica messaggi su un topic viene chiamato \textbf{publisher}, mentre un nodo che legge i messaggi da un topic 
viene chiamato \textbf{subscriber}.
\begin{figure}[H]
    \centering
    \includegraphics*[width=0.75\textwidth]{images/topic_graph.png}
    \caption{Schema di comunicazione}
    \label{fig:topic_graph}
\end{figure}
Per pubblicare un messaggio su un topic bisogna utilizzare l'apposita funzione \verb|publish()| passandole come parametro 
il messaggio che vogliamo pubblicare.
A discapito del nome, questa funzione non pubblica effettivamente il messaggio, ma lo mette in una coda d'attesa (la cui 
dimensione viene specificata quando viene istanziato il publisher).
Un thread separato si occupa di inviare effettivamente il messaggio al topic per renderlo visibile a tutti i nodi subscriber 
connessi.
Se il numero di messaggi in coda supera la sua dimensione, i messaggi pi\'{u} vecchi verranno cancellati per fare spazio 
a quelli pi\'{u} recenti. Quando arriva un nuovo messaggio al subscriber, esso viene salvato 
in una coda d'attesa (stesso funzionamento di quella del publisher) fino a quando ROS non d\'{a} la possibilit\'{a} al nodo 
di eseguire la funzione di callback. Tale funzione \'{e} definita dall'utente e si occupa di processare il messaggio ricevuto. 
ROS eseguir\'{a} una callback solo quando gli verr\'{a} dato il permesso di farlo. Ci sono due modi per farlo:
\begin{itemize}
    \item \verb|ros::spinOnce()| chiede a ROS di eseguire tutte le callback in sospeso restituendo poi il controllo all'utente
    \item \verb|ros::spin()| chiede a ROS di attendere e di eseguire tutte le callback in sospeso fino a quando il nodo non 
          viene spento. \'{E} equivalente a:
          \begin{verbatim}
            while (ros::ok()) {
                ros::spinOnce();
            }
          \end{verbatim} 
          \verb|ros::ok()| ritorna 0 quando: 
          \begin{itemize}
            \item il nodo viene spento attraverso un SIGINT (Ctrl-C) oppure dalla chiamata di \verb|ros::shutdown()| in 
                  un altro punto del codice
            \item un altro nodo con lo stesso nome viene eseguito
          \end{itemize}
\end{itemize}
In altre parole \verb|ros::spin()| vincola il nodo a rimanere sempre e solo in attesa di leggere nuovi messaggi. 
Se il nodo non deve solamente eseguire le callback, allora un loop con \verb|ros::spinOnce()| \'{e} la scelta corretta. 
