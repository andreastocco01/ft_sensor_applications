Per poter eseguire codice ROS serve un ambiente che permetta l'organizzazione e l'utilizzo di tutti i pacchetti necessari. 
Al riguardo, \textbf{catkin} \`{e} il sistema di compilazione ufficiale di ROS che consente la creazione di un workspace per 
organizzare e gestire le applicazioni. 
Un \textbf{pacchetto} ROS contiene al suo interno codici sorgenti, librerie e dati di configurazione. 
I termini `pacchetto' e `applicazione' sono intercambiabili e possono essere utilizzati in modo equivalente.
Per creare un pacchetto \`{e} sufficiente eseguire il comando 
\verb|catkin_create_pkg <nome_pacchetto>| all'interno della cartella \verb|src/| situata nel workspace catkin.
Una volta creato tale workspace \cite{catkin_ws}, il codice sorgente contenuto all'interno dei pacchetti potr\`{a} essere 
compilato ed eseguito.
