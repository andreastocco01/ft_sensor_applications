Per poter eseguire codice ROS serve un ambiente che permetta l'organizzazione e l'utilizzo di tutti i pacchetti necessari.
Al riguardo, \textbf{catkin} \'{e} il sistema di compilazione ufficiale di ROS che consente la creazione di un workspace per 
organizzare e gestire le applicazioni.
I termini `pacchetto' e `applicazione' sono interscambiabili e possono essere utilizzati in modo equivalente (una definizione pi\'{u} 
rigorosa verr\'{a} data successivamente nel corso di questo capitolo).
Una volta creato il workspace catkin \cite{catkin_ws}, il codice sorgente contenuto all'interno dei pacchetti potr\'{a} essere 
compilato ed eseguito.
