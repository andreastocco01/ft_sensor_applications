I topic e i service sono due modalit\`{a} di comunicazione molto diverse a livello concettuale.
Con un topic si instaura una comunicazione asincrona in cui tutti i subscriber connessi attendono la pubblicazione 
di un messaggio da parte del publisher, mentre con un service \`{e} il `server' che rimane in attesa di richieste da 
parte dei client connessi. Questo rende i topic pi\`{u} adatti per la trasmissione di flussi di dati continui (come quelli 
provenienti dai sensori) e i service pi\`{u} indicati nel caso di servizi puntuali, come la richiesta di un calcolo o di un'altra 
specifica azione ad un altro nodo \cite{srv_topic_example}.
Nel corso di questa tesi verranno utilizzate entrambe le modalit\`{a} di comunicazione, con prevalenza di quella topic.